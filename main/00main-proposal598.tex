% \documentclass[sigconf,authordraft]{acmart}
\documentclass[sigconf, nonacm]{acmart}
\AtBeginDocument{%
  \providecommand\BibTeX{{%
    \normalfont B\kern-0.5em{\scshape i\kern-0.25em b}\kern-0.8em\TeX}}}
\usepackage{graphicx}
% \setcopyright{acmcopyright}
% \copyrightyear{2018}
% \acmYear{2018}
% \acmDOI{XXXXXXX.XXXXXXX}
\usepackage{xcolor}
\usepackage{multirow}
\usepackage{graphicx}
\usepackage{multirow}
\usepackage{graphicx}

\definecolor{neonfuchsia}{rgb}{1.0, 0.25, 0.39}
\newcommand{\yiren}[1]{{\small\textcolor{neonfuchsia}{\bf [*** Yi-Ren: #1]}}}
\newcommand{\liri}[1]{{\textsf{\textcolor{teal}{[From Liri: #1]}}}}
%% These commands are for a PROCEEDINGS abstract or paper.
% \acmConference[Conference acronym 'XX]{Make sure to enter the correct
%   conference title from your rights confirmation emai}{June 03--05,
%   2018}{Woodstock, NY}

% \acmPrice{15.00}
% \acmISBN{978-1-4503-XXXX-X/18/06}



\begin{document}


\title{Deep Entity Matching with Knowledge-augmented Pre-trained Language Models}
% ; Preparation affect Entity resolution with Foundation Models Results

% \author{Lan Li \quad Yiren Liu}
% \authornote{Both authors contributed equally to this research.}
% \email{}
% \orcid{}
% \author{}
% \authornotemark[1]
% \email{}
% \affiliation{%
%   \institution{}
%   \streetaddress{}
%   \city{}
%   \state{}
%   \country{USA}
%   \postcode{}
% }


% \author[1]{Lan Li}
% \author[2]{Yiren Liu}

% \email[1]{lanl2@illinois.edu}
% \email[2]{yirenl2@illinois.edu}


\author{
  Lan Li \\ 
 lanl2@illinois.edu
  \and
  Yiren Liu \\
  yirenl2@illinois.edu
}

% \affiliation{%
%   \institution{}
%   \city{}
%   \country{}
% }

% \author{Lars Th{\o}rv{\"a}ld}
% \affiliation{%
%   \institution{The Th{\o}rv{\"a}ld Group}
%   \streetaddress{1 Th{\o}rv{\"a}ld Circle}
%   \city{Hekla}
%   \country{Iceland}}
% \email{larst@affiliation.org}

% \renewcommand{\shortauthors}{Trovato and Tobin, et al.}



Entity resolution is an important and well-studied task for decades, while it is challenging and expensive to apply the traditional way of matching entities with long sequences of textual data and a large dataset, nor hard to compare the data without recognizing the resourceful semantic data information without domain knowledge. 

In this study, we propose a novel framework for incorporating external knowledge into pre-trained language model for entity resolution. We discussed the results of utilizing different knowledge augmentation and prompting methods to improve entity resolution performance, and we assume a deeper language understanding of the entity resolution problems. 

% name not define 
According to our experiment results, our model achieves better results than Ditto, the existing state-of-the-art entity resolution method. 



% \begin{CCSXML}
% <ccs2012>
%  <concept>
%   <concept_id>10010520.10010553.10010562</concept_id>
%   <concept_desc>Computer systems organization~Embedded systems</concept_desc>
%   <concept_significance>500</concept_significance>
%  </concept>
%  <concept>
%   <concept_id>10010520.10010575.10010755</concept_id>
%   <concept_desc>Computer systems organization~Redundancy</concept_desc>
%   <concept_significance>300</concept_significance>
%  </concept>
%  <concept>
%   <concept_id>10010520.10010553.10010554</concept_id>
%   <concept_desc>Computer systems organization~Robotics</concept_desc>
%   <concept_significance>100</concept_significance>
%  </concept>
%  <concept>
%   <concept_id>10003033.10003083.10003095</concept_id>
%   <concept_desc>Networks~Network reliability</concept_desc>
%   <concept_significance>100</concept_significance>
%  </concept>
% </ccs2012>
% \end{CCSXML}

% \ccsdesc[500]{Computer systems organization~Embedded systems}
% \ccsdesc[300]{Computer systems organization~Redundancy}
% \ccsdesc{Computer systems organization~Robotics}
% \ccsdesc[100]{Networks~Network reliability}


% \keywords{}

\maketitle
% \textcolor{blue}{Abstract and Introduction (What topics are you going to work on? Why do you want to work on that topic? What results are you expected to get?)
% Related work (What related papers are you going to read on this topic?)
% Potential solutions and dataset (What are the possible approach and dataset you want to explore on?)
% Reference List}
\section{Introduction}
% PLMs rarely used in data cleaning, in particular, entity resolution tasks
%  challenge 1: missing domain knowledge to help prepare data;
%  challenge 2: missing semantic data types/ a better language understanding of the textual data
%  challenge 3: "too much knowledge incorporation may divert the sentence
% from its correct meaning, which is called knowledge
% noise (KN) issue." \cite{liu_k-bert_2020}
Recent studies using \emph{transformer-based Pre-trained Language Models} (PLMs) have shown their strong ability to perform various types of NLP tasks \cite{min_recent_2021}. However, few studies have discussed the application of PLMs in the domain of data cleaning~\cite{li_deep_2020,narayan_can_2022,vos2022towards}.  
Entity resolution is a common data cleaning task that aims to identify the entries referring to the same real-world entities within or across databases~\cite{christen_data_2012}. 
% \bl{Entity resolution is a common data cleaning task ...}


% Entity resolution, also known as entity matching, record linkage, or data deduplication, is a classical problem in data integration~\cite{zhao_auto-em_2019}. 
% The typical process of entity resolution contains data preparation, data indexing or blocking, and data matching. %  challenge 1: missing domain knowledge to help prepare data;

% Prior work has shown that data 
% Prior work has shown that performing data cleaning before applying ML-based models for entity resolution can improve task performance,  e.g., tokenization or stemming, for threshold-based classifiers \cite{koumarelas_data_2020}.
% \bl{This sentence needs to be re-written for clarity.}.  
% However, it still remains under-studied how data preparators can be applied to datasets from different domains or different types. \bl{it is still an open problem how data preparatory can be applied ... and different data types} For instance,



% \bl{Most ...... the same schema ...., however, in many situations, data from different sources are heterogeneous and use different schema.(heterogeneous schema}

Most existing techniques on entity resolution assume the same schema for records from different sources \cite{elmagarmid_duplicate_2007}. However, in many situations, raw records are obtained from heterogeneous sources and use different schema \cite{enriquez_entity_2017, arabnia_when_2021}. In addition, source data often cover varied domains (e.g., publications, online products, musicians) and in different formats (e.g., numerical, textual, geolocations). 
% , which adds more difficulties for entity resolution tasks 
%Christian (2012) mentioned that schema standardization of records is an essential data preparation for entity resolution task~\cite{christen_data_2012}. 
All of these increase the difficulty for practitioners to perform entity resolution tasks without prior knowledge of the domain-specific information about the data.
% Hulsebos et al \cite{hulsebos_sherlock_2019} introduced Sherlock, a multi-input deep neural network for detecting semantic data types, in which they match 78 semantic types from DBpedia to column headers.  Similarly, Doduo \cite{suhara_annotating_2022}, a multi-task learning framework that is based on
% \emph{pre-trained language models} (PLMs) can predict column types and column relations. 
Thus, we hypothesize that enhancing the external knowledge at the schema and entity level can improve entity resolution tasks.  

With transformer-based PLMs, recent studies draw increasing attention to entity resolution problems~\cite{li_deep_2020, trabelsi_dame_2022}. However, current studies show that the performance might not be ideal when simply inputting the serialized entity pairs into PLMs for classification. Ditto \cite{li_deep_2020} injects domain information: pre-defined entity types (i.e., PRODUCT and NUM), and standardizes the numerical formats to improve the performance before feeding the serialized entity pairs into PLMs.

We push this idea further by injecting more external knowledge at the schema and entity level. Knowledge injection at the schema level aims to infer the fine-grained semantic types (e.g., ALBUM, ARTIST, PUBLISHER) for each column based on data values. 
For the entity level, entity mentions are identified from WikiData and annotated in the initial text with semantic type information of the linked entities. In addition, different formats used to inject external knowledge into the initial entity pairs may vary the performance of PLMs. Thus, three prompting methods are further explored in this study: space, slash, and additional position encoding of PLMs.
% soft position encoding with the visible matrix. 

% Contributions:?
To summarize, starting from state-of-the-art method Ditto \cite{li_deep_2020}, we propose a framework for  \textbf{K}nowledge \textbf{A}ugmented \textbf{E}ntity \textbf{R}esolution (KAER):
\begin{itemize}
    \item using \textbf{C}olumn \textbf{S}emantic \textbf{T}ype (CST) inference and \textbf{E}ntity \textbf{L}inking (EL) in order to inject domain-specific information as additional signals to pre-trained language models. 
    \item leveraging three prompting methods to better augment the acquired knowledge to PLMs.
    \item analyzing the effectiveness of different knowledge injection and prompting methods on entity resolution tasks from different domains and data types.
\end{itemize}
%\vspace{-0.3cm}

% \section{Research Problems}
% In this project, we target the following research questions:
% \begin{itemize}
%     \item (RQ$_1$) How to incorporate domain knowledge into input data when using PLMs for data cleaning?
    
%     \item (RQ$_2$) To what extent does the incorporation of domain knowledge benefit downstream data cleaning tasks, such as entity resolution? 
% \end{itemize}


\section{Related Work}
\subsection{Data Preparation: a Preceding Stage of Entity Resolution}

Data quality impacts the cost and performance of entity resolution systems significantly. Even for an error-free system with perfectly clean data, data from multiple sources are often not saved in a consistent way or carefully controlled for quality \cite{elmagarmid_duplicate_2007}. Usually, the data preparation stage includes parsing, data transformation, and standardization steps, and the goal is to improve the quality of the in-flow data and make the data comparable, and more usable \cite{elmagarmid_duplicate_2007}. 

Even though adjusting and selecting data preparators to prepare data before doing the entity resolution is not the focus of this paper, we will still apply some general data transformations to deal with specific data types. Data transformations are used to convert the data into the correct type or format that conforms to their domains \cite{elmagarmid_duplicate_2007}. Enhancing data quality by preprocessing data also contributes to the result of Sherlock \cite{hulsebos_sherlock_2019}, which can detect the semantic data types. Here we mainly use general transformations to help normalize the atomic types of data, such as \textit{string}, \textit{integer}, \textit{Boolean}. For instance, the \textit{float} type of data values for $Year$ is meaningless. So a simple conversion of the data from \textit{float} to \textit{integer} is required. Meanwhile, encoding issues also appear across the data values that need to be addressed at this stage. For numerical data, range checking could be used to ensure that data is in the appropriate range \cite{elmagarmid_duplicate_2007}. Preprocessing composite data values will be more complicated that requires a few steps, i.e., we usually first split the data values and then prepare data values separately, and finally merge the cleaned data values.

\subsection{Schema Matching: The Context of Entity Resolution}
% Assigned to Lan Li
Entity resolution is to identify and match records that refer to the same real-world entity. In many situations, raw records are stored from heterogeneous sources, while most existing techniques on entity resolution predefine the same schema for records from different sources \cite{elmagarmid_duplicate_2007}. Moreover, records from different sources might follow different structures at the attribute level, which adds more difficulties for entity resolution tasks \cite{enriquez_entity_2017, arabnia_when_2021}. Therefore, schema matching is required to discuss before we talk about entity resolution tasks. In particular, we consider the schema matching task as the context of processing entity resolution for schema matching guides which two records should be paired at the column level before processing the entity resolution at the row level. \cite{lin_efficient_2020} perform schema matching by creating a \textit{full schema} in which they compare and merge records from different sources into \textit{super record} before the entity resolution. In this way, they avoid information loss during the schema-matching process.

The main operation in manipulating schema information is \textit{Match}, which inputs two schemas and produces a mapping between two elements of the two schemas that relate semantically to each other \cite{rahm_survey_2001}. \cite{rahm_survey_2001} select the matching algorithms (a.k.a matchers) based on the application domain and schema types, and use data types to constrain the search space of correspondences. 

\subsection{Pretrained Model for Entity Resolution}
% Assigned to Lan Li

\cite{zhao_auto-em_2019} propose a transfer-learning approach to entity matching (EM), leveraging pre-trained entity matching models that are based on large-scale, production knowledge bases (KB). Auto-EM enables entity type detection and entity matching at the attribute level by learning a large amount of data from KBs. In this way, they achieve a high EM quality with little labeled training data. \cite{wu_zeroer_2020} introduce ZeroER that requires \textit{Zero} labeled examples for entity resolution task. They leverage a powerful generative model based on Gaussian Mixture Models for learning the match and mismatch distributions. 

Unlike Recurrent Neural Network (RNN) used in (\cite{zhao_auto-em_2019}, \cite{mudgal_deep_2018}), a few recent work applies transformer-based PLMs to entity resolution tasks. Paganelli et al. analyzes that simply fine-tuning BERT for matching/non-matching classification task mainly updates the last layers of the BERT components and BERT can recognize the input sequence represents a pair of records~\cite{paganelli_analyzing_2022}. Li et al. leverages siamese network structure to PLMs in order to improve the efficiency of PLMs during blocking phase~\cite{li_improving_2021}. 
Ditto by \cite{li_deep_2020} is now the state-of-the-art entity matching system based on pre-trained Transformer-based language models. In addition, Ditto provides deeper language understanding for entity resolution by injecting domain knowledge, summarizing the key information, and augmenting with more difficult examples for training data. The method of conditioning the language model, a.k.a, "prompting," is a hot topic in Nature Language Processing \cite{kojima_large_2022}. They \cite{kojima_large_2022} propose Zero-shot-CoT, a single zero-shot prompt that evokes a chain of thought from large language models, highlighting that the performance of the language model has been affected by different templates of the prompt. 
% Auto-EM introduce the framework to pretrain an RNN-based attribute type classifier on large knowledge base .
\vspace{-0.5em}
\subsection{Knowledge augmentation}
There are existing studies about methods that can be used to inject external domain knowledge into PLMs: 

% \subsection{Semantic data types recognition at column-level}

\textbf{Semantic data types detection at column-level.}
Sherlock \cite{hulsebos_sherlock_2019} is a novel system that uses a deep learning approach to detect semantic data types at the column level. Sherlock predicts the conceptual domain of a column when there is no schema or the existing schema cannot provide a fine-grained description of the data. Doduo~\cite{suhara_annotating_2022} is another recent method for annotating  columns with semantic types and for adding semantic relations between columns. Both Sherlock and Doduo methods can be used to inject domain-specific knowledge for columns with existing names or missing names.


\textbf{Entity Linking:}
Entity Linking \cite{li_deep_2020} refers to the task of linking entity mentions appearing in natural language text with their corresponding entities in an external knowledge base, e.g., Wikidata. 

% Existing work using traditional methods for entity linking .... \yiren{TODO}

Existing study has explored using PLMs to perform entity linking tasks, and achieved promising results. 
Zhang et al.'s work \cite{zhang_ernie_2019} introduced the model ERNIE by jointly pre-training BERT with a masked language modeling objective. The knowledge augmented BERT model is found to outperform existing methods in multiple knowledge-driven tasks. 
Later work by Peter et al. \cite{peters_knowledge_2019} futher improved the ERNIE model by introducing a trainable entity linker module and alignment between entity embedding and BERT embedding. 
Recent work by Ayoola et al. \cite{ayoola_refined_2022} introduced an entity linking method by using fine-tuning a pre-trained language model over Wikipedia data. The model has shown strong ability in zero-shot domain adaptation, which is used as a baseline method for entity linking in our study. 

\textbf{Knowledge Incorporation} %\liri{Liri}
Recent works shows that injecting external knowledge to PLMs can benefit Natural Language Reprocessing (NLP) downstream tasks~\cite{zhang_ernie_2019,peters_knowledge_2019,liu_k-bert_2020,wang_k-adapter_2021, wang_kepler_2021}. Furthermore, the method to inject the identified external knowledge into PLMs matters. Liu et al. propose the model K-BERT adding soft-position encoding and visible matrix to the augmented input sequence to avoid exposing too much knowledge to the original input and corresponding models~\cite{liu_k-bert_2020}. Another direction of knowledge incorporation is to align the knowledge embedding and PLMs output into same embedding space. The work from Wang et al. \cite{wang_kepler_2021} introduced a joint pre-training method based on roBERTa to map knowledge base entity and natural language entity description into the sample space. The knowledge embedding produced by KEPLER can be utilized as an additional input information source in our entity resolution task. K-Adaptor, proposed by Wang et al.~\cite{wang_k-adapter_2021}, stacks pre-trained multiple adapter model with RoBERTa for different types of knowledge, e.g. one for factual knowledge from knowledge base and another for linguistic-related syntax knowledge. The effectiveness of these methods in incorporating knowledge into entity resolution tasks can be further explored.

\section{Potential Solutions and Dataset}
% \section{Experience, Readiness, Usage Plans and Funding Source(s)}\label{sec-experience}
We plan to examine several potential methods that can improve the task performance of entity resolution. 
First, we want to investigate whether the incorporation of additional external knowledge (e.g., WikiData) can help improve PLMs' ability in handling entity resolution tasks; Second, we want to propose a new method to better incorporate useful information as a signal for entity resolution.

We will experiment on thirteen open-access datasets used by DeepMatcher~\cite{mudgal_deep_2018} and Ditto~\cite{li_deep_2020}. We summarize the statistics of these datasets in Table~\ref{tab:exp-data}.


\begin{table}[h]
\centering
\resizebox{\columnwidth}{!}{%
\begin{tabular}{|c|l|c|c|c|c|}
\hline
\textbf{Data Type}          & \multicolumn{1}{c|}{\textbf{Datasets}} & \textbf{Domains} & \textbf{Size} & \textbf{\# Positive} & \textbf{\# Attributes} \\ \hline
\multirow{7}{*}{Structured} & Amazon-Google                          & software         & 11,460        & 1,167                & 3                      \\ \cline{2-6} 
                            & Walmart-Amazon                         & electronics      & 10,242        & 962                  & 5                      \\ \cline{2-6} 
                            & DBLP-Scholar                           & citation         & 28,707        & 5,347                & 4                      \\ \cline{2-6} 
                            & DBLP-ACM                               & citation         & 12,363        & 2,220                & 4                      \\ \cline{2-6} 
                            & iTunes-Amazon                          & music            & 539           & 132                  & 8                      \\ \cline{2-6} 
                            & BeerAdvo-RateBeer                      & beer             & 450           & 68                   & 4                      \\ \cline{2-6} 
                            & Fodors-Zagats                          & restaurant       & 946           & 110                  & 6                      \\ \hline
\multirow{2}{*}{Textual}    & Abt-Buy                                & product          & 9,575         & 1,028                & 3                      \\ \cline{2-6} 
                            & Company                                & company          & 112,632       & 28,200               & 1                      \\ \hline
\multirow{4}{*}{Dirty}      & iTunes-Amazon                          & music            & 539           & 132                  & 8                      \\ \cline{2-6} 
                            & DBLP-ACM                               & citation         & 12,363        & 2,220                & 4                      \\ \cline{2-6} 
                            & DBLP-Scholar                           & citation         & 28,707        & 5,347                & 4                      \\ \cline{2-6} 
                            & Walmart-Amazon                         & electronics      & 10,242        & 962                  & 5                      \\ \hline
\end{tabular}%
}
\caption{Datasets summary}
\label{tab:exp-data}
\end{table}
We have created and implemented the experimental framework, and performed several initial experiments as a proof of concept. The preliminary results are encouraging and seem to support our hypothesis, i.e.,  that introducing additional knowledge can improve the performance of using PLMs for entity resolution. Additionally, we also find that additional domain knowledge introduced has exhibited a stronger improvement over ER tasks with a relatively limited amount of available training data (as in Table 1, iTunes-Amazon).
	
% Please add the following required packages to your document preamble:
% \usepackage{graphicx}
\begin{table}[!h]
\resizebox{\columnwidth}{!}{%
\begin{tabular}{l|ccc}
\hline
                       & \multicolumn{3}{c}{\textbf{Datasets}}                                                                                                                                                                                         \\ \hline
                       & \multicolumn{1}{c|}{\textbf{\begin{tabular}[c]{@{}c@{}}DBLP-GoogleScholar \\ (N=17,223)\end{tabular}}} & \multicolumn{1}{c|}{\textbf{\begin{tabular}[c]{@{}c@{}}Abt-Buy \\ (N=5,743)\end{tabular}}} & \textbf{\begin{tabular}[c]{@{}c@{}}iTunes-Amazon\\ (N=321)\end{tabular}} \\ \hline
Ditto                  & \multicolumn{1}{c|}{0.944}                               & \multicolumn{1}{c|}{0.824}                                                               & 0.625                                                                    \\ \hline
Ditto + EL & \multicolumn{1}{c|}{0.946}                               & \multicolumn{1}{c|}{0.831}                                                               & 0.720                                                                    \\ \hline
\end{tabular}%
}
\caption{Preliminary experimental results (F1 scores): entity linking exhibits more improvements over smaller datasets.}
\label{tab:exp-results}
\end{table}
% Table 1. preliminary experiment results (F1 scores); Entity linking exhibits better improvement over smaller dataset;



% % What is entity matching/resolution/deduplication problem? 

% % why care DC: it is difficult to capture the extra steps/massage the data before applying the novel algorithms. 

% The ability to repeat and compare the experimental results is fundamental aspect of the research, however, experimenters "massage" the data before applying their novel algorithms without explicitly record them \cite{koumarelas2020data}. In other words, the implicit using data cleaning or data preparation before the entity resolution task result in experiment results incomparable. 

% In this paper, we aim to survey how the data preprocessing performs in entity resolution with foundation models tasks. 
% % TODO 
% Challenges:

% Quite a few papers have not explicitly describe how they preprocess the data even though they list some data quality issues. 
% - \cite{mudgal2018deep} propose three kinds of entities
% \section{Background}
% In paper \cite{li2020deep}, they implement \emph{DITTO}, a novel entity matching system based on pretrained language models. \emph{DITTO} has improved the matching quality and outperforms the previous state-of-the-art. The optimizations of \emph{DITTO} can be categorized as data preparation include: 1. it injects domain knowledge; 2. it summarizes long pieces of input and only keep the most essential information.

% In paper \cite{zhao2019auto}, 





%%
%% The acknowledgments section is defined using the "acks" environment
%% (and NOT an unnumbered section). This ensures the proper
%% identification of the section in the article metadata, and the
%% consistent spelling of the heading.
% \begin{acks}
% To Robert, for the bagels and explaining CMYK and color spaces.
% \end{acks}

%%
%% The next two lines define the bibliography style to be used, and
%% the bibliography file.
\bibliographystyle{ACM-Reference-Format}
\bibliography{Reference}

%%
%% If your work has an appendix, this is the place to put it.
% \appendix

% \section{Research Methods}

% \subsection{Part One}

% Lorem ipsum dolor sit amet, consectetur adipiscing elit. Morbi
% malesuada, quam in pulvinar varius, metus nunc fermentum urna, id
% sollicitudin purus odio sit amet enim. Aliquam ullamcorper eu ipsum
% vel mollis. Curabitur quis dictum nisl. Phasellus vel semper risus, et
% lacinia dolor. Integer ultricies commodo sem nec semper.

% \subsection{Part Two}

% Etiam commodo feugiat nisl pulvinar pellentesque. Etiam auctor sodales
% ligula, non varius nibh pulvinar semper. Suspendisse nec lectus non
% ipsum convallis congue hendrerit vitae sapien. Donec at laoreet
% eros. Vivamus non purus placerat, scelerisque diam eu, cursus
% ante. Etiam aliquam tortor auctor efficitur mattis.

% \section{Online Resources}

% Nam id fermentum dui. Suspendisse sagittis tortor a nulla mollis, in
% pulvinar ex pretium. Sed interdum orci quis metus euismod, et sagittis
% enim maximus. Vestibulum gravida massa ut felis suscipit
% congue. Quisque mattis elit a risus ultrices commodo venenatis eget
% dui. Etiam sagittis eleifend elementum.

% Nam interdum magna at lectus dignissim, ac dignissim lorem
% rhoncus. Maecenas eu arcu ac neque placerat aliquam. Nunc pulvinar
% massa et mattis lacinia.

\end{document}
\endinput
%%
%% End of file `sample-authordraft.tex'.
