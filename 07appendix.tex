\paragraph{A. Hyperparameter Settings.} 
Baseline and KAER models share the same hyperparameter setting, in which the batch size equals 64, the max length equals 512, the learning rate is 3e-5, and the number of training epochs is 20. 
%Other parameters keep unchanged with the default setting from Ditto. 
%-lr", type=float, default=3e-5)
%    parser.add_argument("--n_epochs", type=int, default=20)

\paragraph{B. Baseline models.} 
\begin{table}[!b]
\centering
\begin{tabular}{|l|l|}
\hline
Dataset                  & Injection       \\ \hline
dirty/DBLP-Google        & Ditto + General \\ \hline
dirty/iTunes-Amazon      & Ditto + Product \\ \hline
structured/iTunes-Amazon & Ditto + Product \\ \hline
structured/Amazon-Google & Ditto + Product \\ \hline
structured/DBLP-ACM      & Ditto + General \\ \hline
structured/DBLP-Google   & Ditto + General \\ \hline
textual/Abt-Buy          & Ditto + Product \\ \hline
\end{tabular}
\caption{Datasets and Corresponding Ditto default knowledge injection methods}
\label{tab:dittoinject}
\end{table}
This paper includes two baseline models, i.e., RoBERTa without any knowledge injection and Ditto with its default knowledge injection methods. 

In detail, Table~\ref{tab:dittoinject} shows the datasets and corresponding injection methods utilized by Ditto. "+ General" indicates that Ditto injects seven entity types into the corresponding dataset. The seven types include 'PERSON', 'ORG', 'LOC', 'PRODUCT', 'DATE', 'QUANTITY', and 'TIME'. "+ Product" indicates that Ditto annotates as 'PORODUCT', any entity mentions in the following types, i.e., 'NORP', 'GPE', 'LOC', 'PERSON', and 'PRODUCT'.

\paragraph{C. Two-sided Paired T-test.}
The two-sided paired T-test is implemented to test whether the predictions of different models are significantly different. 
